\section{Background}
This section will present the theory behind the current economic system based on Fractional Reserve Banking (FRB), 
proposed economic systems based on 100\% Reserve Banking (100RB), and the decentralized economic system based on blockchain and cryptography. 
The section will then discuss their individual issues and compare them to each other.


\subsection{The current system: Fractional Reserve Banking (FRB)}
All economies in the world use different configurations of the Fractional Reserve Banking (FRB) system. FRB is a system in which commercial banks hold only a fraction of their customers' deposits as reserves at any time. These reserves provide liquidity to satisfy short-term customer withdrawals, short-term obligations, and settle interbank transactions. Commercial banks wants to earn profit on the deposit they hold, so they invest the rest of the deposit in finanical investments such as loans, bonds, or stock. This system allows commercial banks to serve as financial intermediaries between savers and borrowers, where savers get interest on their deposits, borrowers get loans, and commercial banks earns interest and profit from the investments. Commercial banks serve an important function in the modern economy by facilitating this efficient capital allocation, promoting economic growth, and facilitating money transactions. However, this system also carries inherent risks, such as bank runs and financial instability, which necessitate the implementation of regulations and safeguards trough monetary policy by central banks and fiscal policy from authorities to maintain financial stability. 

The money multiplier theory is a simple way to illustrate how commercial banking works. The formula is given by $\frac{1}{Reserve Ratio}$. A Reserve Ratio of 10\% gives a money multiplier of $\frac{1}{0.1}=10$. In this system banks could turn a \$1000 initial depoist into \$10000 by lending out \$900 of the initial depoist keeping \$100 in reserve. The borrower spends the \$900, and the recipient deposits it back into the bank. The bank then lends out \$810 of this deposit keeing \$90 in reserve, and the process continues. However this concept is criticized for beeing misleading as several countries does not set a legal reserve requirement \cite{chicagorevisited}, including the USA after the covid pandemic started in 2020 \cite{FRBinvestopedia}. 

There are two schools of toughts on money creation. Endogenous money theory posits that money supply is driven by the demand for loans and credit within the economy, with commercial banks creating money through lending. The central bank's role is to accommodate credit demand by adjusting interest rates. Exogenous money theory asserts that money supply is determined by central banks and governments through open market operations, quantitative easing, and printing physical cash. Commercial banks can create money within the limits set by the central bank. 

\textcite{mcleay2014money} from the central bank of england argue that the former (endegenous) theory is much more prevelalent. Commercial banks are the creators of money. The authors argue that for the money multiplier theory to hold, the amount of reserves must be a binding constraint on lending, and the central bank must directly determine the amount of reserves, however in modern economies (in most cases), central banks do no control the quanity of reserves, but rather implement monetary policy by setting the price of reserves by setting the interest rate. \textcite{chicagorevisited} argue that when commercial banks wants to hold more reserves, the central bank will oblige by giving out a reserve loans to the commerical bank. In effect, this has the consequence that commerical banks can in theory create infinite money trough infinite loans. In practice, this unique privilage is constrained by the profitability on loans and competition in the market. For example, commercial banks have to pay interest on the reserve loans, this interest rate ripples into borrowers loan, where difference on interest rate of the central bank reserve loans and borrowers loans is the profit for commercial bank. Too high rate and borrowers do not want to take on new loans. The central banks interest rate is set by monetary policy.  This web of incentives is much more complex and complicated than can be described here, however it illustrates the issue that commercial banks can missuse their power if they find it profitable. In theory, this power is also limited by regulation and oversight, however regulation usually lags behind the market, and oversight is often very limited by small government budgets, lobbying, and other factors. As an example of where banks missued their power, lets take a look at the great recession of 2008.

In XXXX commercial banks realized that they could pacakge many individual mortages into a well diversified security. They could then earn a lot of money by selling this security, backed by mortages, as a low-risk investement. Not a bad idea, but as time went on, commercial banks and the shadow banking system, continously wanted increased profits on this system, issuing increaslingy risky mortages and manipulating rating agencies to classify the securities to give them the top rating. Then the shadow banking system made the problem much worse by placing massive and risky bets on these securites. The limitations set by regulation and oversight did not succeed in identifiying or stopping this problem neither did the web of incentives and reserve requirements (that the USA had at the time). (source)

When components of the FRB system fail, the government, central banks, or other relevant entites step in to stop the failure before it can cascade into a catastrophich systemtic breakdown. \textcite{berger2020tarp} argues that the cost of bailouts is of a relativly low percentage compared to the potential cost of letting the crisis play out. However, since entites, such as commercial banks, know that there is a high probability they will recieve a bailout in the event of failure, they are incentivized to increase their apetite for risk to take on riskier and riskier investments for greater profit. The so called \textit{too big to fail} commerical banks are even more incentivized to increase their risk, because a failure of these banks would cause a catastropich systematic breakdown alone, and the governemnt or related entites will have no option but to bail them out.

\subsection{Maturity transformation}
Maturity transformation is a key element of our current economic system that matches short-term funding to long-term loans. Long-term loans provide people and businesses financing to perform economic development. It would be increadibly risky for people to lock up all of their funds for the duration (maturity) of the long-term loans, they might need it! Maturity transformation is the act of combining short-term funding, for example from checking accounts or saving accoutns, to long-term loans so that people can withdraw and use their money, but still finance these important long-term loans, at the cost of financial stability and the possibility of bank-runs. 

One option to this problem is to enable the sale of the long-term loan so that people can withdraw and use their money. (todo expand)


\textcite{Stellinga2021} argue that maturity transformation is an important economic function and 

\subsection{Economic theory}
Quantity Theory of Money (QTM) is an economic theory that suggests there is a direct relationship between the supply of money in an economy and the price level of goods and services. It argues that changes in the money supply will lead to proportional changes in the price level. The theory assumes that the velocity of money (the rate at which money is spent) and the level of real output are constant in the short run. \textcite{fisherQTM} defined the formula for the quantity theory of money as:
\begin{equation}
    M \times V = P \times Y
\end{equation}

Where M is the money supply, V is the velocity of money, P is the price level, and Y is real output. The equation states that the total amount of spending in an economy (MV) is equal to the total value of goods and services produced (PY).

Monetarism is an economic theory that emphasizes the importance of controlling the money supply to manage inflation and stabilize the economy. The theory is mostly associated with the nobel price winning economist Milton Friedman in his work \textcite{friedman2008monetary}. Monetarists argue that central banks should focus on maintaining a steady growth rate of the money supply, rather than using discretionary fiscal policies. Thus, monetarism advocates for a rules-based approach to monetary policy, where central banks consistently and predictably adjust the money supply to achieve economic stability. The k-percent rule proposed by \textcite{friedman2008monetary} states that money supply (M) should grow at a constant annual rate tied to the growth of nominal gross domestic product. 

Monetarists argue the money supply (M) drives the QTM equation. Essentially, alterations in M directly impact and dictate employment, inflation (P), and output (Y). \textcite{friedman2008monetary} assumed that velocity (V) remained constant, but monetarists today consider V to be readily predictable instead \cite{monetarismInvestopedia}. 


\subsection{Mortages and debt backed securites}

Commercial banks issue mortages to borrowers and can then sell these mortages (at a discount??) to an investement company that packages and pools individual mortages into Mortage-Backed Securites (MBS). These securites have a fixed interest rate and monthly payouts based on the repayment of the underlying loans. In this system, commercial banks act as a financial intermediary that lend investors money to homebuyers. 

\textcite{MBSInvestopedia} states that this system works well if all parties do what they are supposed to. The commercial banks grant mortages based on reasonable standards, homeowner pays on time, and the credit rating agencies that review MBS perform proper due dilligence and assign a true rating based on the underlying loans and risk. However, obviously this process failed in 2007-8, where commercial banks gave mortages to anyone and everyone without due dilligence or proper risk assesments, homeowners who should not have recieved mortages could not pay their mortages and the mortages eventually defaulted, and credit rating agencies was incentivized or manipulated to not perform their due diligence and give the securites the highest rating even tough the underlying mortages was, in the words of the hit movie, The Big Short, \textit{dogshit} \cite{thebigshort}.

% How many loans are sold into MBS? Does banks hold loans anymores? If they don't, how does this affect the creation of money?
There are two types of MBS:
\begin{itemize}
    \item \textbf{Pass-Troughs}: A security where mortage payments are simply collected and passed on to the investors, typically with a specific maturity of 5, 15, or 30 years, depending on the velocity of repayements.
    \item  \textbf{Collateralized Mortage Obligations (CMO)}: A sligthly more complex security that consist of multiple securites, or tranches, that have  different maturities, yields (profits), risk, and priority of repayment on default.
\end{itemize}

Advantages and disadvantages of MBS according to \textcite{MBSInvestopedia}:
\begin{itemize}
    \item \textbf{Attractive yield}: MBS pay a fixed interest rate that is usually higher than U.S. government bonds, and pays out each month, where other bonds have other structures, such as one single payment at maturity 
    \item \textbf{Safe Investments}: MBS are considered relativly low-risk. If the MBS is guaranteed by the government or otherwise insured, investors do not have to worry about defaults. Since an MBS is well diversifed with multiple mortages, the risk is diversified. 
    \item \textbf{Detached from the rest of the market}: There is a relativly low correlation between MBS and corporate bonds or the stock market.
    \item \textbf{Prepayment risk}: If borrowers pay off their loans early or refinance their loans it may negativly impact returns. 
    \item \textbf{Interst rate risk}: If interest rates increase, new amount of new mortages decrease, causing the housing market to decrease, and value of the MBS will drop.
\end{itemize}

A Collateralized Debt Obligation (CDO) can be seen as a generalization of MBS/CMO where the underlying assets is not only mortages, but any cash-flow generating assets, such as automobile loans, credit cards, and aircraft leases, in different tranches (differnt maturites, yields, and risks).

Syntethic CDOs are bets or wagers put on the performance of a CDO, essentially leveraging the CDO multiple times for greater profit or greater loss. Synthetic CDOs on CDOs of MBSs, underneath it all backed by \textit{"dogshit"} mortages, was one of the major cause of the great recession of 2008, and these synthetic CDOs was aptly named, again by the big short movie, \textit{"dogshit wrapped in catshit"}. 

Assed-Backed Securites (ABS) are collaterlized and backed by any kind of assets, usually debt assets that generate cash-flow at a steady rate. This security is a further generalization of CDOs. 

MBS, CMO, CDO, and ABS are in the authors opinion, a generally good idea to pool together underlying assets or debts to diversify risk into simple instruments, however it is clear that this system was massivly perverted in the 2000s, especially with derivitates on top of these securites, such as synthetic CDOs that lead to the great recission of 2008.

\subsection{The great recession of 2008}

Bailout: The Federal Reserve bought \$4.5 trillion of MBS \cite{MBSInvestopedia}.


\subsection{The oncomming storm of 2023}
Assets become increasingly centralized in a few banks. A collapse in one of these banks...

\subsection{100\% Reserve Banking}
In contrast, a 100\% Reserve Banking (100RB) system requires banks to hold the entire amount of their customers' deposits as reserves. In general, this means that banks cannot create new money through lending and act more like safe deposit institutions. The 100\% reserve banking system can reduce the risk of bank runs and financial crises but may also constrain credit creation and economic growth.

Several approaches towards a 100RB system exist:

\begin{itemize}
    \item The Chicago Plan \cite{fisher1936} developed by economists during the 1930s, proposes separating monetary and credit functions of the banking system by requiring 100\% reserves for deposits and centralizing money issuance. This aims to reduce bank runs, financial crises, and inflation risks while allowing more effective monetary policy management. The plan's key features include a 100\% reserve requirement, separating credit and money creation, and government control of money issuance. Critics argue that the plan may constrain credit creation and economic growth. \textcite{fisher1936} claimed the following advantages:
    \begin{enumerate}
        \item Much better control of a major source of business cycle fluctuations, sudden increases and contractions of bank credit and of the supply of bank-created money.
        \item Complete elimination of bank runs.
        \item Dramatic reduction of the (net) public debt.
        \item Dramatic reduction of private debt, as money creation no longer requires simultaneous debt creation.
        \end{enumerate}
        \item The Chicago Plan Revisited \cite{chicagorevisited} from International Monetary Fund (IMF) in 2012 revisits the idea within the modern economy of the US and finds support for all of \textcite{fisher1936} claims. Additionally, output gains approach 10\%, and steady-state inflation can drop to zero.
        \item 2018 Swiss citizens’ (popular) initiative, \textit{‘For crisis-safe money: Money creation by the National Bank only! (Sovereign Money Initiative)’} \cite{SwizzInitiative}. The initiative was defeated with 442k yes votes and 1379k no votes. The Federal Council and Parliament were against the initiative due to no precedent in any country for this type of system, a radical departure from the current system, which they claim to work well and has taken effective measures to improve financial stability, and that the national bank would receive an undesirable level of power.
        \item In the aftermath of the 2008 financial crisis in Iceland, a study on monetary and banking reform \cite{IcelandReport} was commissioned in 2015 by the prime minister. A Sovereign Money System was strongly considered, but no further action was taken.
\end{itemize}

\subsection{Sovereign Money System (SMS)}
\textcite{Stellinga2021} brlilianty summarizes the history, workings, and different proposals of the Soverign Money System. This section will summarize this work, without citing specifics works. TODO expand this section to find and argue for the best alternative.

In a Soverign Money System, all money is held at a central bank or in entites where deposits are 100\% backed by reserves at the central bank or government bonds. Entites must legally and financially seperate payment accounts with customers deposit from other activites, such as investing. 

Configuration options 
\begin{itemize}
    \item Commerical banks have a full 100\% reserve in central banks on deposits.
    \item Commercial banks have a full 100\% reserve in liquid aseets, such as government bonds, on deposits.
    \item Customers deposits into a payment account at the central bank.
    \item Financed by debt, where customers place deposit into an investment account, and the commercial bank can then invest these funds in for example loans. The loans can still have a money multiplier effect, and customers can still withdraw their deposits before the underlying assets have reached maturity. This configuration is therefor succeptible to many of the issues of FRB, and can generally be seen as a more slow-moving FRB system.
    \item Financed by equity, where entites sell shares to raise funds to invest, essentially becoming investment funds. All shareholders will share profits or losses. If a shareholder wants to withdraw or use their money, they can sell their share. 
    \item \textcite{chicagorevisited} allow for the option that commerical banks can be financed by central banks trough loans however other proposals are more cautious of this configuration.   
    \item Money creation is directly shifted to the central bank instead of commerical banks in the current system. 
\end{itemize}

In the SMS system, banks cannot create deposits out of loans out of thin air, but have to raise the money. 

Kotlikoff (44, TODO add ref), argues that a new supervisory body, a Finanical Authority, should examine and rate all financial instriumts to asses risks. 

\subsection{The economy of cryptocurrencies and other Decentralized Digital Assets (DDA)}
One of the primary tenets behind cryptocurrencies, tokens, stablecoins and other variants of Decentralized Digital Assets (DDA) in general is having full control of your own money. Most DDAs are not invested or otherwise put to use and therefore align more to the 100RB and SMS system over the current FRB system, in-fact, the author (todo argue) would argue that DDAs can be described as a Self-Soverign Money (SSM) system that share most of the similarities, advantages, and disadvantages of the 100RB/SMS system. The 100RB/SMS system is designed for a single country or economy, while SSM is inherently a global system. The SSM system can therefor be seen as an extension and a complement to national 100RB/SMS systems.

\subsection{Central Bank Digital Currency (CBDC)}
CBDC may be an implementation of 100RB. However, the fed stated that the issuance of CBDC may be limited per person to not draw money out from the banking system... CBDCs may not align with the current FRB system, or if it does, the goal must probably be to replace cash with digial cash, aka only a small percentage of total money (2-3̭\%??). 