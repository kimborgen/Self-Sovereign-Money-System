Our current Fractional Reserve Banking (xRB) system relies heavily on maturity transformation, that is, matching short-term savings to fund long-term debts. An important economic function but one that also causes bank runs and financial instability. Alternatives such as 100\% Full Reserve Banking (FRB), Sovereign Money System (SMS), and cryptocurrencies, address this issue, among many other issues of xRB, but have been criticized for increasing the price and decreasing the availability of loans, stunting economic growth severely. This working paper presents the theory and an implementation prototype of a fully (Self) Sovereign Money System, applicable to both sovereign ownership in individual countries and a decentralized and globally distributed economy. The system has a native currency, a Self-Sovereign Token (SST), interchangeable with a Central Bank Digital Currency (CBDC). Loans are enabled by pooling together savers' SST and creating new SST if the saver's funds cannot alone meet the demand for new loans. In exchange, the savers, including the system itself, receive proportional ownership of the pool of all loans, represented by a Pooled Debt Token (PDT). This token is a pass-through that distributes repayments with interest to the holders. The system will burn all the repayments it receives to mitigate the inflationary effect. The value of this pool in the native currency SST is autonomously, continuously, and fully transparently calculated based on all the underlying variables in each loan, solving the major issues with its sibling, Collateralised Debt Obligations, that sparked the great recession of 2008. When savers want to use or withdraw their savings, they sell the PDT to the system, which creates new SSTs based on the calculated value of PDT. Thus removing the possibility of a bank-run scenario and always enabling instant liquidity of the Pooled Debt Contracts. Lastly, the interest rate in this system is also autonomously, continuously, and fully transparently calculated to influence the rate of new loans and repayment of existing loans based on a formula designed to target a stable economic growth rate of nominal Gross Domestic Product (GDP) of 3\%.  