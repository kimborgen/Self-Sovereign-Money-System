\documentclass{article}
%\usepackage{cite}
\usepackage[hidelinks]{hyperref}
\usepackage[parfill]{parskip}
\usepackage{biblatex}
\usepackage[utf8]{inputenc}
\usepackage{babel,csquotes,xpatch}% recommended


\title{[DRAFT Working Paper] Towards a (Self) Sovereign Money System with equivalent economic growth to Fractional Reserve Banking}
\author{Kim Aksel Tahuil Borgen}
\date{\today}
\addbibresource{thesis.bib}
\begin{document}
\maketitle
\begin{abstract}
    Our current Fractional Reserve Banking (xRB) system relies heavily on maturity transformation, that is, matching short-term savings to fund long-term debts. An important economic function but one that also causes bank runs and financial instability. Alternatives such as 100\% Full Reserve Banking (FRB), Sovereign Money System (SMS), and cryptocurrencies, address this issue, among many other issues of xRB, but have been criticized for increasing the price and decreasing the availability of loans, stunting economic growth severely. This working paper presents the theory and an implementation prototype of a fully (Self) Sovereign Money System, applicable to both sovereign ownership in individual countries and a decentralized and globally distributed economy. The system has a native currency, a Self-Sovereign Token (SST), interchangeable with a Central Bank Digital Currency (CBDC). Loans are enabled by pooling together savers' SST and creating new SST if the saver's funds cannot alone meet the demand for new loans. In exchange, the savers, including the system itself, receive proportional ownership of the pool of all loans, represented by a Pooled Debt Token (PDT). This token is a pass-through that distributes repayments with interest to the holders. The system will burn all the repayments it receives to mitigate the inflationary effect. The value of this pool in the native currency SST is autonomously, continuously, and fully transparently calculated based on all the underlying variables in each loan, solving the major issues with its sibling, Collateralised Debt Obligations, that sparked the great recession of 2008. When savers want to use or withdraw their savings, they sell the PDT to the system, which creates new SSTs based on the calculated value of PDT. Thus removing the possibility of a bank-run scenario and always enabling liquidity of the Pooled Debt Contracts. Lastly, the interest rate in this system is also autonomously, continuously, and fully transparently calculated to influence the rate of new loans and repayment of existing loans based on a formula designed to target a stable economic growth rate of nominal Gross Domestic Product (GDP) of 3\%.  
\end{abstract}

\subsection{Acronyms}

\begin{tabular}{ll}
\textbf{Acronym} & \textbf{Definition} \\
FRB & Fractional Reserve Banking \\
100RB & 100\% Reserve Banking \\
\end{tabular}

\section{Background}
This section will present the theory behind the current economic system based on Fractional Reserve Banking (FRB), 
proposed economic systems based on 100\% Reserve Banking (100RB), and the decentralized economic system based on blockchain and cryptography. 
The section will then discuss their individual issues and compare them to each other.


\subsection{The current system: Fractional Reserve Banking (FRB)}
All economies in the world use different configurations of the Fractional Reserve Banking (FRB) system. FRB is a system in which commercial banks hold only a fraction of their customers' deposits as reserves at any time. These reserves provide liquidity to satisfy short-term customer withdrawals, short-term obligations, and settle interbank transactions. Commercial banks wants to earn profit on the deposit they hold, so they invest the rest of the deposit in finanical investments such as loans, bonds, or stock. This system allows commercial banks to serve as financial intermediaries between savers and borrowers, where savers get interest on their deposits, borrowers get loans, and commercial banks earns interest and profit from the investments. Commercial banks serve an important function in the modern economy by facilitating this efficient capital allocation, promoting economic growth, and facilitating money transactions. However, this system also carries inherent risks, such as bank runs and financial instability, which necessitate the implementation of regulations and safeguards trough monetary policy by central banks and fiscal policy from authorities to maintain financial stability. 

The money multiplier theory is a simple way to illustrate how commercial banking works. The formula is given by $\frac{1}{Reserve Ratio}$. A Reserve Ratio of 10\% gives a money multiplier of $\frac{1}{0.1}=10$. In this system banks could turn a \$1000 initial depoist into \$10000 by lending out \$900 of the initial depoist keeping \$100 in reserve. The borrower spends the \$900, and the recipient deposits it back into the bank. The bank then lends out \$810 of this deposit keeing \$90 in reserve, and the process continues. However this concept is criticized for beeing misleading as several countries does not set a legal reserve requirement \cite{chicagorevisited}, including the USA after the covid pandemic started in 2020 \cite{FRBinvestopedia}. 

There are two schools of toughts on money creation. Endogenous money theory posits that money supply is driven by the demand for loans and credit within the economy, with commercial banks creating money through lending. The central bank's role is to accommodate credit demand by adjusting interest rates. Exogenous money theory asserts that money supply is determined by central banks and governments through open market operations, quantitative easing, and printing physical cash. Commercial banks can create money within the limits set by the central bank. 

\textcite{mcleay2014money} from the central bank of england argue that the former (endegenous) theory is much more prevelalent. Commercial banks are the creators of money. The authors argue that for the money multiplier theory to hold, the amount of reserves must be a binding constraint on lending, and the central bank must directly determine the amount of reserves, however in modern economies (in most cases), central banks do no control the quanity of reserves, but rather implement monetary policy by setting the price of reserves by setting the interest rate. \textcite{chicagorevisited} argue that when commercial banks wants to hold more reserves, the central bank will oblige by giving out a reserve loans to the commerical bank. In effect, this has the consequence that commerical banks can in theory create infinite money trough infinite loans. In practice, this unique privilage is constrained by the profitability on loans and competition in the market. For example, commercial banks have to pay interest on the reserve loans, this interest rate ripples into borrowers loan, where difference on interest rate of the central bank reserve loans and borrowers loans is the profit for commercial bank. Too high rate and borrowers do not want to take on new loans. The central banks interest rate is set by monetary policy.  This web of incentives is much more complex and complicated than can be described here, however it illustrates the issue that commercial banks can missuse their power if they find it profitable. In theory, this power is also limited by regulation and oversight, however regulation usually lags behind the market, and oversight is often very limited by small government budgets, lobbying, and other factors. As an example of where banks missued their power, lets take a look at the great recession of 2008.

In XXXX commercial banks realized that they could pacakge many individual mortages into a well diversified security. They could then earn a lot of money by selling this security, backed by mortages, as a low-risk investement. Not a bad idea, but as time went on, commercial banks and the shadow banking system, continously wanted increased profits on this system, issuing increaslingy risky mortages and manipulating rating agencies to classify the securities to give them the top rating. Then the shadow banking system made the problem much worse by placing massive and risky bets on these securites. The limitations set by regulation and oversight did not succeed in identifiying or stopping this problem neither did the web of incentives and reserve requirements (that the USA had at the time). (source)

When components of the FRB system fail, the government, central banks, or other relevant entites step in to stop the failure before it can cascade into a catastrophich systemtic breakdown. \textcite{berger2020tarp} argues that the cost of bailouts is of a relativly low percentage compared to the potential cost of letting the crisis play out. However, since entites, such as commercial banks, know that there is a high probability they will recieve a bailout in the event of failure, they are incentivized to increase their apetite for risk to take on riskier and riskier investments for greater profit. The so called \textit{too big to fail} commerical banks are even more incentivized to increase their risk, because a failure of these banks would cause a catastropich systematic breakdown alone, and the governemnt or related entites will have no option but to bail them out.

\subsection{Maturity transformation}
Maturity transformation is a key element of our current economic system that matches short-term funding to long-term loans. Long-term loans provide people and businesses financing to perform economic development. It would be increadibly risky for people to lock up all of their funds for the duration (maturity) of the long-term loans, they might need it! Maturity transformation is the act of combining short-term funding, for example from checking accounts or saving accoutns, to long-term loans so that people can withdraw and use their money, but still finance these important long-term loans, at the cost of financial stability and the possibility of bank-runs. 

One option to this problem is to enable the sale of the long-term loan so that people can withdraw and use their money. (todo expand)


\textcite{Stellinga2021} argue that maturity transformation is an important economic function and 

\subsection{Economic theory}
Quantity Theory of Money (QTM) is an economic theory that suggests there is a direct relationship between the supply of money in an economy and the price level of goods and services. It argues that changes in the money supply will lead to proportional changes in the price level. The theory assumes that the velocity of money (the rate at which money is spent) and the level of real output are constant in the short run. \textcite{fisherQTM} defined the formula for the quantity theory of money as:
\begin{equation}
    M \times V = P \times Y
\end{equation}

Where M is the money supply, V is the velocity of money, P is the price level, and Y is real output. The equation states that the total amount of spending in an economy (MV) is equal to the total value of goods and services produced (PY).

Monetarism is an economic theory that emphasizes the importance of controlling the money supply to manage inflation and stabilize the economy. The theory is mostly associated with the nobel price winning economist Milton Friedman in his work \textcite{friedman2008monetary}. Monetarists argue that central banks should focus on maintaining a steady growth rate of the money supply, rather than using discretionary fiscal policies. Thus, monetarism advocates for a rules-based approach to monetary policy, where central banks consistently and predictably adjust the money supply to achieve economic stability. The k-percent rule proposed by \textcite{friedman2008monetary} states that money supply (M) should grow at a constant annual rate tied to the growth of nominal gross domestic product. 

Monetarists argue the money supply (M) drives the QTM equation. Essentially, alterations in M directly impact and dictate employment, inflation (P), and output (Y). \textcite{friedman2008monetary} assumed that velocity (V) remained constant, but monetarists today consider V to be readily predictable instead \cite{monetarismInvestopedia}. 


\subsection{Mortages and debt backed securites}

Commercial banks issue mortages to borrowers and can then sell these mortages (at a discount??) to an investement company that packages and pools individual mortages into Mortage-Backed Securites (MBS). These securites have a fixed interest rate and monthly payouts based on the repayment of the underlying loans. In this system, commercial banks act as a financial intermediary that lend investors money to homebuyers. 

\textcite{MBSInvestopedia} states that this system works well if all parties do what they are supposed to. The commercial banks grant mortages based on reasonable standards, homeowner pays on time, and the credit rating agencies that review MBS perform proper due dilligence and assign a true rating based on the underlying loans and risk. However, obviously this process failed in 2007-8, where commercial banks gave mortages to anyone and everyone without due dilligence or proper risk assesments, homeowners who should not have recieved mortages could not pay their mortages and the mortages eventually defaulted, and credit rating agencies was incentivized or manipulated to not perform their due diligence and give the securites the highest rating even tough the underlying mortages was, in the words of the hit movie, The Big Short, \textit{dogshit} \cite{thebigshort}.

% How many loans are sold into MBS? Does banks hold loans anymores? If they don't, how does this affect the creation of money?
There are two types of MBS:
\begin{itemize}
    \item \textbf{Pass-Troughs}: A security where mortage payments are simply collected and passed on to the investors, typically with a specific maturity of 5, 15, or 30 years, depending on the velocity of repayements.
    \item  \textbf{Collateralized Mortage Obligations (CMO)}: A sligthly more complex security that consist of multiple securites, or tranches, that have  different maturities, yields (profits), risk, and priority of repayment on default.
\end{itemize}

Advantages and disadvantages of MBS according to \textcite{MBSInvestopedia}:
\begin{itemize}
    \item \textbf{Attractive yield}: MBS pay a fixed interest rate that is usually higher than U.S. government bonds, and pays out each month, where other bonds have other structures, such as one single payment at maturity 
    \item \textbf{Safe Investments}: MBS are considered relativly low-risk. If the MBS is guaranteed by the government or otherwise insured, investors do not have to worry about defaults. Since an MBS is well diversifed with multiple mortages, the risk is diversified. 
    \item \textbf{Detached from the rest of the market}: There is a relativly low correlation between MBS and corporate bonds or the stock market.
    \item \textbf{Prepayment risk}: If borrowers pay off their loans early or refinance their loans it may negativly impact returns. 
    \item \textbf{Interst rate risk}: If interest rates increase, new amount of new mortages decrease, causing the housing market to decrease, and value of the MBS will drop.
\end{itemize}

A Collateralized Debt Obligation (CDO) can be seen as a generalization of MBS/CMO where the underlying assets is not only mortages, but any cash-flow generating assets, such as automobile loans, credit cards, and aircraft leases, in different tranches (differnt maturites, yields, and risks).

Syntethic CDOs are bets or wagers put on the performance of a CDO, essentially leveraging the CDO multiple times for greater profit or greater loss. Synthetic CDOs on CDOs of MBSs, underneath it all backed by \textit{"dogshit"} mortages, was one of the major cause of the great recession of 2008, and these synthetic CDOs was aptly named, again by the big short movie, \textit{"dogshit wrapped in catshit"}. 

Assed-Backed Securites (ABS) are collaterlized and backed by any kind of assets, usually debt assets that generate cash-flow at a steady rate. This security is a further generalization of CDOs. 

MBS, CMO, CDO, and ABS are in the authors opinion, a generally good idea to pool together underlying assets or debts to diversify risk into simple instruments, however it is clear that this system was massivly perverted in the 2000s, especially with derivitates on top of these securites, such as synthetic CDOs that lead to the great recission of 2008.

\subsection{The great recession of 2008}

Bailout: The Federal Reserve bought \$4.5 trillion of MBS \cite{MBSInvestopedia}.


\subsection{The oncomming storm of 2023}
Assets become increasingly centralized in a few banks. A collapse in one of these banks...

\subsection{100\% Reserve Banking}
In contrast, a 100\% Reserve Banking (100RB) system requires banks to hold the entire amount of their customers' deposits as reserves. In general, this means that banks cannot create new money through lending and act more like safe deposit institutions. The 100\% reserve banking system can reduce the risk of bank runs and financial crises but may also constrain credit creation and economic growth.

Several approaches towards a 100RB system exist:

\begin{itemize}
    \item The Chicago Plan \cite{fisher1936} developed by economists during the 1930s, proposes separating monetary and credit functions of the banking system by requiring 100\% reserves for deposits and centralizing money issuance. This aims to reduce bank runs, financial crises, and inflation risks while allowing more effective monetary policy management. The plan's key features include a 100\% reserve requirement, separating credit and money creation, and government control of money issuance. Critics argue that the plan may constrain credit creation and economic growth. \textcite{fisher1936} claimed the following advantages:
    \begin{enumerate}
        \item Much better control of a major source of business cycle fluctuations, sudden increases and contractions of bank credit and of the supply of bank-created money.
        \item Complete elimination of bank runs.
        \item Dramatic reduction of the (net) public debt.
        \item Dramatic reduction of private debt, as money creation no longer requires simultaneous debt creation.
        \end{enumerate}
        \item The Chicago Plan Revisited \cite{chicagorevisited} from International Monetary Fund (IMF) in 2012 revisits the idea within the modern economy of the US and finds support for all of \textcite{fisher1936} claims. Additionally, output gains approach 10\%, and steady-state inflation can drop to zero.
        \item 2018 Swiss citizens’ (popular) initiative, \textit{‘For crisis-safe money: Money creation by the National Bank only! (Sovereign Money Initiative)’} \cite{SwizzInitiative}. The initiative was defeated with 442k yes votes and 1379k no votes. The Federal Council and Parliament were against the initiative due to no precedent in any country for this type of system, a radical departure from the current system, which they claim to work well and has taken effective measures to improve financial stability, and that the national bank would receive an undesirable level of power.
        \item In the aftermath of the 2008 financial crisis in Iceland, a study on monetary and banking reform \cite{IcelandReport} was commissioned in 2015 by the prime minister. A Sovereign Money System was strongly considered, but no further action was taken.
\end{itemize}

\subsection{Sovereign Money System (SMS)}
\textcite{Stellinga2021} brlilianty summarizes the history, workings, and different proposals of the Soverign Money System. This section will summarize this work, without citing specifics works. TODO expand this section to find and argue for the best alternative.

In a Soverign Money System, all money is held at a central bank or in entites where deposits are 100\% backed by reserves at the central bank or government bonds. Entites must legally and financially seperate payment accounts with customers deposit from other activites, such as investing. 

Configuration options 
\begin{itemize}
    \item Commerical banks have a full 100\% reserve in central banks on deposits.
    \item Commercial banks have a full 100\% reserve in liquid aseets, such as government bonds, on deposits.
    \item Customers deposits into a payment account at the central bank.
    \item Financed by debt, where customers place deposit into an investment account, and the commercial bank can then invest these funds in for example loans. The loans can still have a money multiplier effect, and customers can still withdraw their deposits before the underlying assets have reached maturity. This configuration is therefor succeptible to many of the issues of FRB, and can generally be seen as a more slow-moving FRB system.
    \item Financed by equity, where entites sell shares to raise funds to invest, essentially becoming investment funds. All shareholders will share profits or losses. If a shareholder wants to withdraw or use their money, they can sell their share. 
    \item \textcite{chicagorevisited} allow for the option that commerical banks can be financed by central banks trough loans however other proposals are more cautious of this configuration.   
    \item Money creation is directly shifted to the central bank instead of commerical banks in the current system. 
\end{itemize}

In the SMS system, banks cannot create deposits out of loans out of thin air, but have to raise the money. 

Kotlikoff (44, TODO add ref), argues that a new supervisory body, a Finanical Authority, should examine and rate all financial instriumts to asses risks. 

\subsection{The economy of cryptocurrencies and other Decentralized Digital Assets (DDA)}
One of the primary tenets behind cryptocurrencies, tokens, stablecoins and other variants of Decentralized Digital Assets (DDA) in general is having full control of your own money. Most DDAs are not invested or otherwise put to use and therefore align more to the 100RB and SMS system over the current FRB system, in-fact, the author (todo argue) would argue that DDAs can be described as a Self-Soverign Money (SSM) system that share most of the similarities, advantages, and disadvantages of the 100RB/SMS system. The 100RB/SMS system is designed for a single country or economy, while SSM is inherently a global system. The SSM system can therefor be seen as an extension and a complement to national 100RB/SMS systems.

\subsection{Central Bank Digital Currency (CBDC)}
CBDC may be an implementation of 100RB. However, the fed stated that the issuance of CBDC may be limited per person to not draw money out from the banking system... CBDCs may not align with the current FRB system, or if it does, the goal must probably be to replace cash with digial cash, aka only a small percentage of total money (2-3̭\%??). 

\section{Deriving a new system}
The investor is in this context a person or entity that decided to invest their money by lending it out, for example trough a checking account, term-locked deposit accounts, investment accounts, or equity in a investment fund, in exchange either directly or inderectly for a debt contract from the borrower.

The Fractional Reserve Banking system has resulted in enourmous economic growth, however it has a lot of fundemental and deep issues.
\begin{itemize}
    \item Commercial banks leverage customers deposits by only keeping a fraction of the customers deposits in reserve and investing the rest (trough loans, bonds, or stocks). Essentially gambling customers deposits. If the investment decreases, customers may end up loosing their deposits, and a bank-run may be triggered. (In the case of SVB). 
    \item Even in a "well-functioning" bank, a bank-run can be triggered at any time, for example by external forces such as a bank collapse on the other side of the world. Causing a domino cascade that may lead to a systematic collapse. 
    \item Deposit insurance is meant to mitigate these issues, but only goes up to \$250k in the US and €250k in Europe, more than enough for most people, but a drop in the ocean for businesses. 
    \item A bank-run may require the bank to sell their investments early, often with a discount, taking on more losses, further worsening the situation.
    \item We are beginning to see a centralizing of money into very few, extremly large banks.
    \item Commercial banks have the theoretical power to create infinite money trough infinite loans, if they determine it is profitable. This power is only limited by a web of incentices on this profitabillity, for example setting the interest rate that the commerical bank has to pay on their reserves at the central bank. This power is also limited by regulation and oversight, that often is not a sufficient tool to limit the industry. 
    \item Because commercial banks are limited by profitability more than rules, they may be incentivized to find new ways to increase their own profitability at the cost of increased systematic risk. For example when commercial banks started to give out as many mortages as possible without any restrictions in the runup to the great recission of 2008, they did so because the mortages, and the risk they have, could be sold to other banks or financial entites almost immediatly, for instant profit. 
    \item Because bail-outs, bail-ins or other types of support from the relevent authorites is a necessicity in this system to stop the issues from cascading into a systematic collapse incentivizes commercial banks and other entites to take on increasingly more risk. Especially the "too big to fail" banks are incentivized to take on massive risk for greater profit because they have to be bailed out, there may not be any other options. 
\end{itemize}

Issues with MBS and CDOs.

Issues with SMS

Issues with crypto/DDAs/SSM

(Self-)Soverign Money Systems address many of the issues of Fractional Reserve Banking by moving more to a 100\% Reserve Banking system. The main driver of the economic growth is the price and availability of credit (loans). (S)SMS have been criticied for increased prices and decreased availability of credit 

Any system with loans will have a money multiplier effect. Even if the loans come 100\% from equity. The recepients of the money from loans can downstream decide to become investors with this money, leading to more loans. The current system relies on maturity transformation, the effect where investors can withdraw their money before the investment has matured, leading to bank-runs and instabilites. It is therefor crucial to put sufficient restrictions and effort into design on this system so that investors cannot cause a bank-run and collapse the system. An effective way to ensure this cannot happen is to no longer rely on maturity transformation and to bind the funds to the maturity of the contract, or in other words, an investor cannot withdraw funds from the system as long as those funds are locked in the financial contracts. If the investor wants to withdraw funds from the system they will have to sell their position in the financial contract.

So lets build our case:

The system must create loans to borrowers. How this happens and types of loans is defered. A loan creates a debt contract with the borrower. This PDCtract debt contract encapsulates every aspect of the loan, including repayements, interest rate, risk, collateral, value of collateral, and every legal document. This debt contract will always have a value based on the info and state in the debt contract. Since this debt contract has a unique value derived by its internal variables, it may have a different value than other similar debt contracts, making it a Non-Fungible Debt Contract. The formule of the value is given by

\begin{equation}
    v_{DC} = f(X_{DC} = x_{1}, x_{2}, ..., x_{n})
\end{equation}

Where $V_{DC}$ is the Value of the Debt Contract, $X_{DC}$ is an array of all relevant information $x_{1}, x_{2}, ..., x_{n}$ to the Debt Contract, and $f$ is the process to calculate the value of the debt contract based on all the relevant information. 

This single debt contract is a financial instrument that we can do everything with. It can be used in payements, sold, traded, packaged, or pooled into bigger instruments. If the contract is pooled it will begin to have similarities with Mortage-Backed Securities (MBS) if the debt contracts is mortages and Asset-Backed Securities (ABS) in any other case. For now we will pool all debt contracts into a single variable $PDC$. The value of the $PDC$ is directly given by the value of every debt contract. The formula is given by 

\begin{equation}
    v_{PDC} = f(X_{v_{DC}} = v_{DC_{1}}, v_{DC_{2}}, ..., v_{DC_{n}} )
\end{equation}

A key issue with the mortage crisis that led to the great recession of 2008 was the lack of transparrency at each stage of the system. At the end, investors only looked at the final numbers and ratings without looking at the underlying assets. The people who actually looked at the underlying assets could see all of the final numbers and ratings was \textit{dogshit}, \textcite{thebigshort} made a movie from the perspecive of these people and how they uncovered these lies/frauds and took out a short position on what common public sense determined to be the most secure asset class.

One problem is the discretionary power each element in this system has and how disconnected all numbers are. The underlying info and resulting value of the debt contract was not transmitted further up the chain to the end securities and their ratings and values. Each element in the chain had discretionary powers to set the value, info, and rating they deemed accurate. An incompetent, dishonest, or malicious actor in only one of the elements in this chain will result in the wrong value, rating, and info of the end security. 

A single autonoums system should exist to record, track, and manage all relevant info and numbers. All underlying information is put in this system and aggregate values, ratings, and information should be automatically created from this underlying information trough rules, formulas, and other clearly defined processes. Such a system may have three fundamental weaknesses, (1) the creators, managers, and operators of the system can manipulate any part it, (2) actors can manipulate or otherwise submit bad processes, and (3) an actor can submit bad underlying information.

The first isssue (1) can be addressed with a fully open, transparrent, and secure system on an un-manipulatable computing and storage substrate. Blockchain and other Distributed Ledger Technologies present such a substrate. If you can represent the entire system in code and data, and submit this code and data to a DLT, both the code and data is replicated and distributed to every participant of the system, in permissionless and open DLT this can mean everyone in the world. Any participant can see and verify that the system is designed according to the specification, the code is openly executed, and the system verifies that it was executed correctly, and any participant can verify that the code was executed correctly. 

The second issue (2) can be partly addressed by the openness of DLT, anyone can analyse, scrutinize, and verify the process, however this does not mean the the process is objectivly "good". This can be addressed with standardization, simplification, and limitations of what the system does. However, it may be wrong to issue limitations on what can be created, so it may be best to rely on standardization as a communication tool to investors, but allow non-standardized processes, with clear marks.

The third issue (3) is a bit more complex to address. These actors must be legally obligated under strict, explicit, and broad rules to provide accurate and timly information. A strong incentive policy should also exist, however in contrasts to current systems web of incentives that may lag behind the real incentives, legal obligations should be stronger in jurdistrictions with strong and functioning legal systems. In jurdistrictions with questionable legal systems, incentives should be stronger. The actors should be placed under supervision and oversight of a relevant authority, which may employ for example regular audits or other tools to discover incompetent, dishonest, or otherwise malicious behaviour. Please note that the current scope of this paper does not include legal aspect, so it will focus on incentives only, but any real-life system must include strong legal guarantess. 

We now have a system which can issue and manage individual loans and pool them into well diversified securities. 

How much loans should be issued? According to the Quantity Theory of Money and monetarist economic school of tought the money supply in an economy should target the nominal GDP, lets say 3\%. That is that the growth of money creation ($\overrightarrow{M}$) times the growth of velocity of money ($\overrightarrow{V}$) should equal 3\%. For the time beeing we can assume that $\overrightarrow{V}$ remains constant at 0. (We may look at adjusting the formula to respond to changes in $\overrightarrow{V}$ later.). In the current monetary system money creation comes primarly from new loans \cite{chicagorevisited}. In this scenario we assume that a super majority of money creation comes only from this system, and ignore other sources of money creation, so we must investigate how to keep money creation within this system alone to the GDP growth exactly, 3\%. 

Targeting nominal GDP is only one option, meant to keep the system as stable as possible and potentially eliminting business cycles. However money creation can also be set to target the actual demand for new loans. This might align more to the problem statement of this project, to allow for the same price and availability of loans as the current system. Economic schools of tought is not an exact science, and may even not apply to a system based on 100RB, so targeting nominal GDP growth on money creation alone may not work. More resarch and deep analysis is required! But that is out-of-scope for now. 

\textcite{chicagorevisited} argued for a 100\% Reserve Baking system where new loans are funded both by equity and by loans from the central bank that creates the funds for this loan directly. The authors analysed this scenario in depth based on modern economic models of the U.S. economy and found support for all of the original claims in the chigago plan in the 1930s \cite{fisher1936}, and even a 10\% output gain of the economy. 

In Self-Soverign Money Systems there is controlling entity behind the system, however, we can define any kind of system we want with code. We can therfor create a currency, or a token, where the code can create and destroy tokens according to defined rules. This new token can be called $c$.This allows greater control of the stability of the system.

In this system the funds from the new loans should first come from investors that deposit $c$ into an investment account/contract $i_{c}$, if there isnt enough funds to meet the demand for new loans, we can create new tokens $c$ to fund the remaining loans, where the system itself becomes an investor. Very similar to the role of a central bank in SMS. When the system recieves repayments from debt contracts the repayments must be burned to mitigate the inflation. If all the debt contracts mature, all $c$ that was created, was also destroyed. In an end-game scenario, if the demand of loans remain constant over the long-term, the destroyment of repayment from existing debt contracts should balance out the new funds that are created for new loans. If the demand for loans always grows, as it should in a growing economy, then money is effectivly created, as the rate of new funds created for loans outweight the rate of burn of the repayements. 

In this closed-loop system the monetary supply will be given only by the total supply of the token $c_{ts}$. However, real-world systems are not closed, and $v_{PDC}$ can be used as payment or in exchange of any other assets, goods, or services. In theory, since individual debt contracts have different values, they may be classified as Non-Fungible Contract, however, you can also make a system where you sell ownership in an individual debt contracts or ownership in a larger PDCs, essentially making it fungible. In-fact it is difficult to limit others to create synthetic products on top of your own, so better to make it fungible to limit the popularity of synthetic products. Because of this, $v_{PDC}$ can be used as near-moneis and should be counted as a currency in our analysis. The money supply now is given by the formula

\begin{equation}
    M = c_{ts} + v_{PDC} = c_{0} + c_{NT} - c_{DT} + v_{PDC}
\end{equation}
Where $c_{ts}$ is the total supply of the token c, $c_{0}$ is the intial supply of the token, $c_{NT}$ are all the new tokens issued for loans, and $c_{DT}$ is all tokens that has been destroyed. The growth rate is given by:

\begin{equation}
    \overrightarrow{M} = \overrightarrow{c_{NT}} - \overrightarrow{c_{DT}} + \overrightarrow{v_{PDC}}
\end{equation}

The question is now how to design a system where $\overrightarrow{M}$ targets nominal GDP, around 3\%.
\begin{equation}
    \overrightarrow{M} = GDP = 3\%
\end{equation}

New loans fully backed by investment contracts $i_{c}$ will increase $v_{PDC}$ only. This is because the tokens still circulate in the system, it even may end up in new loans. New loans fully backed by issuing new tokens will grow both $c_{NT}$ and $v_{PDC}$. The growth of these terms can only be balanced with $\overrightarrow{c_{DT}}$. 

An effective and simple way to influence these terms is to create an interest rate $IR$. A higher interest rate $IR$ will decrease the attractivity of new loans, thus lower the amount of new loans, decreasing $\overrightarrow{c_{NT}} + \overrightarrow{v_{PDC}}$. A higer interest rate will also have the positive consequence that it may incentivize a more rapid repayement of the loans, increasing $\overrightarrow{c_{DT}}$. It also has the negative consequence that the risk of default or missing repayements increase with interest rate, thus resulting in lower $\overrightarrow{v_{PDC}}$. A lower interest rate $IR$ has the exact oposite effect. An interest rate seems like a simple and effective tool to incentivize this process, however great care must be taken in such a design. The equation for interest rate can be given by

\begin{equation}
    \overrightarrow{M} = y_{1}(IR) \times \overrightarrow{c_{NT}} - y_{2}(IR) \times \overrightarrow{c_{DT}} + y_{3}(IR) \times \overrightarrow{v_{PDC}}
\end{equation}
Where $y_{n}(IR)$ are functions of the interest rate that determine the effect $IR$ has on the individual growth rates. 

This equation will also depends on the amount that investors are willing to invest at any given moment. As such investments will not grow $c_{NT}$. However to model this equation we need more specific designs. 

Lets define a phased cycle $\tau$. Breaking it down makes it easier to analyse and communicate. 
\begin{itemize}
    \item[$\tau_{0}$] Investors deposit the tokens they want to invest into a shared investment account $SIA$ where the contributions from investors are given by $SIA_{d}$, the d for deposit. At the close of $\tau_{0}$ investors are assigned a percentage $PER_{N=d_{i}}$, where $N=1,2,...,n$ of $SIA_{d}$ according to how much they invested and what the total amount of $SIA_{d}$ was.
    \item[$\tau_{1}$] Qualified Legal Enitites (QLE) collect applications for debt from people or entites. QLEs determines who should recieve loans with what terms, and creates a preliminary Debt Contract $DC_{pre}$ for each loan.
    \item[$\tau_{2}$] The Principal Ammount $PA$ of all preliminary Debt Contracts are summed $PA_{\forall DC_{pre}} = \sum_{0}^{n} PA_{DC_{pre}}$. If the size of $SIA_{i}$ is not large enough to fill this $PA_{PDC_{prelim}}$ the system will create new tokens and invest those into $SIA$ so that the contribution from the system is given by $SIA_{s} =  PA_{PDC_{prelim}} - SIA_{d}$. All the funds from $SIA$ are then assigned to every $DC_{pre}$. And the system then assignes a percentage $PER_{0}$ of $SIA$ to itself with $\frac{SIA_{s}}{SIA - SIA_{i}}$ and investors are assigned their individual percentage by $\frac{SIA_{d} \times PER_{N=d_{i}}}{SIA - SIA_{s}}$. The system and the investors are now fully equal, the size and percentage of their respective investments tracked by the $PRE_{n}$ variable. 
    \item[$\tau_{3}$] All new $DC_{pre}$ are pooled together into a preliminary Pooled Debt Contract $PDC_{pre}$. The value $v_{PDC_{pre}} = f(PDC_{pre})$ is calculated. 
    \item[$\tau_{4}$] The $PDC_{pre}$ is added to the general $PDC$. All $DC_{pre}$ are activated into active Debt Contracts $DC$ and the funds are released to the borrowers. The value of the old PDC increases exactly by the value of the preliminary PDC, given by $v_{PDC_{new}} = v_{PDC_{old}} + v_{PDC_{pre}}$. The investors (including the system) must now be given a percentage of the $PDC$. The percentage of the $PDC$ can tracked in a token. In fact, the $PDC$ itself can be Fungible Token. The Pooled Debt Contract Token (PDCT). The total amount of token increases by $\frac{v_{PDC_{new}}}{v_{PDC_{pre}}}$ and the new PDCTs are distributed according to $PRE$. The old tokenholders are dilluted so that they see no increase nor decrease in their value. This process is similar to what happens when a company fundraises funds in exchange of newly issued stock. 
\end{itemize}

The value might increase even more because the $PDC$ grows larger and gets more diversified (TODO). 

One potential issue that new investors now are investing in all existing debt contracts. And vice versa, old investors now are invested in new debt contracts. The quality of these debt contracts might differ due to the evolving nature of the system. Each debt contract might have different terms, regulations, limitations, implementation of debt contracts, etc. However all of these variables and qualities should be accuratly and automatically calculated into a current value. So all else beeing equal, if a new Debt Contract had a slightly higher quality, it should result in a higer value, and the investor should therefor recieve more tokens of the $PDC$. This should therefor not be an issue. If it is, then an alternative system based on many Pooled Debt Contracts should be explored. For example if $\tau$ is set to a week, and the outcome of this process is a completly new, unique, and fully indipendant $PDC$. 

Even if the $v_{PDC}$ grows up to 3\% in value, as defined in the equation x, investors will not see this increase as they get increasingly dilluted. 













Since this system relies on automatic, open, and transparrent code execution, we can build the interest rate $IR$ into the token smart contract to algorithmically determine the interest rate based on all the underlying variables. 











New loans then come from a combination of $i_{c}$ and



So we can create a new currency within the system. 




Is it possible to create a Fungible Contract? So these contracts can be used as payments? 




. How these loans are issued and managed, and the type of loans, is defered. 

These financial contracts 


 the owners of the debt contract, in th


Who determines the rules of the system
\begin{itemize}
    \item Discretionary power (current system)
    \item Automatic rule-based power
\end{itemize}

Generalized role of central banks in the current system \cite{Stellinga2021}
\begin{itemize}
    \item Inflation stabilized at x=2\%
    \item Price stability
    \item Maximizing employement
    \item Low long-term interest rates
    \item Financial stability
    \item efficient and reliable payemnt system
    \item protecting depositors 
\end{itemize}

The goal of this system should be to to target nominal GDP by targeting a 3\% increase in money supply each year. It should do this with automatic rule-based power within the smart contract. All other functions are shifted to other entites. 



\printbibliography
\end{document}
